%%%%%%%%%%%%%%%%%%%%%%%%%%%%%%%%%%%%%%%%%%%%%%%%%%%%%%%%%%%
% Capitolo 1

\chapter{Contesto}
\label{contesto}

In questo capitolo sarà introdotto il distretto Databenc e l'azienda Ancitel. 


\section{DATABENC}
Il progetto Databenc (\url{http://www.databenc.it}) (Distretto Ad Alta Tecnologia per i BENi Culturali) è nato in Campania, grazie all'Università degli Studi di Napoli Federico II e all'Università di Salerno, con l'intento di stabilire una programmazione strategica per valorizzare i beni culturali, il patrimonio ambientale e il turismo.
Databenc ha l'obiettivo di costituire, tra università, centri di ricerca, imprese e amministrazioni comunali presenti sul territorio, una rete che focalizzi le proprie risorse su di un programma di alta tecnologia al fine di creare nuove realtà imprenditoriali (spin-off, start-up), nuove figure professionali, percorsi di alta formazione qualificati, valorizzazione delle conoscenze (brevetti, know how). 
Il distretto è un contenitore in cui riunire ed integrare itinerari eterogenei di ricerca, formazione ed innovazione, con l'obiettivo comune della tutela e della valorizzazione del patrimonio culturale campano inteso in senso esteso: territori, siti, beni e attività. 
\subsection{Le linee di intervento}
Gli ambiti di intervento del Distretto si sviluppano su tre linee portanti: 
\begin{itemize}
\item Conoscenza integrata: la prima forma di tutela di un bene è nella conoscenza e, per tale motivo, è necessario realizzare un esauriente sistema di salvaguardia cognitiva del patrimonio culturale;
\item Monitoraggio diagnostico: ai fini della tutela di un bene risulta indispensabile il monitoraggio diagnostico inteso in senso ampio, che non si limiti solo alla verifica dell'integrità materiale del bene stesso ma si estenda anche all'area in cui il bene è inserito o alle dinamiche turistiche che lo coinvolgono;
\item Fruizione sostenibile: un aspetto fondamentale del bene culturale è quello del suo utilizzo.  Perciò risulta indispensabile conseguire un utilizzo sostenibile del patrimonio culturale.
\end{itemize}

In sintesi, Databenc ha come obiettivo l'introduzione di una nuova ottica con cui affrontare il grave problema della tutela e della valorizzazione del patrimonio culturale, e la creazione di un sistema che faccia della Campania una regione dell'innovazione e un centro di produzione e diffusione di cultura capace di attrarre capitali economici e, soprattutto, capitali umani.


\section{Ancitel}

Ancitel S.p.A. (\url{http://www.ancitel.it}) è la principale società dell'ANCI - Associazione Nazionale Comuni Italiani - e da 25 anni supporta gli enti locali nella gestione di tutti i processi di innovazione.
Dalla sua fondazione, avvenuta nel 1987,  Ancitel affianca le pubbliche amministrazioni locali con un'ampia rete di servizi e progetti ideati per rispondere alle loro esigenze operative quotidiane.
In quanto partner dei Comuni, Ancitel agisce ed opera ogni giorno come centro di competenza per fornire loro soluzioni e strumenti pensati per facilitarne e supportarne l'azione quotidiana ed affrontare le sfide dell'innovazione. 
Grazie ad una profonda conoscenza delle dinamiche interne alla Pubblica Amministrazione, Ancitel ha conseguito notevoli capacità di ascolto, dialogo ed intervento. Per questo uno dei ruoli fondamentali dell'azienda è quello di promuovere e favorire lo scambio delle informazioni tra gli enti pubblici, centrali e locali.
Grazie alle forti competenze e professionalità e alla capacità di valorizzare le esperienze locali e coinvolgere trasversalmente l'intera rete dei Comuni Italiani, Ancitel è stata scelta come partner affidabile dai principali organismi istituzionali italiani, quali Camera dei Deputati, Presidenza del Consiglio dei Ministri, Ministero dell'Ambiente, Ministero dell'Interno, Ministero del Lavoro e delle Politiche Sociali, Ministero dello Sviluppo Economico, Autorità per l'energia elettrica e il gas.
All'interno del distretto Databenc, Ancitel è uno dei principali soci. 


\section{Fruizione informazioni culturali e turistiche in mobilità}
Nell'ambito dei beni culturali, vi sono diversi progetti che coinvolgono enti locali, grandi, piccole e medie imprese ed istituti universitari.
L'obiettivo comune è quello di sviluppare strumenti di valorizzazione e capitalizzazione dell'offerta culturale e delle risorse ambientali, al fine di promuovere e commercializzare l'offerta turistica da parte delle P.A. locali.
Si rende spesso necessaria la definizione e lo sviluppo di una piattaforma abilitante su cui basare servizi per l'offerta culturale. Una piattaforma che ponga al centro le informazioni da offrire agli utenti, che renda facile ed accessibile la fruizione di esse e che possieda validi strumenti per la conservazione e la salvaguardia.
Le stesse informazioni, inoltre, vanno validate e standardizzate, in modo da consentire facilmente l'estrazione e la catalogazione automatica, l'analisi e la correlazione di esse attraverso motori semantici.
%E' altresì auspicabile l'introduzione di un sistema di feedback del bene culturale, in cui è realizzabile il concetto di esplorazione personalizzata in base ad un'analisi delle esperienze degli utenti sul territorio, per comprendere meglio le aspettative del turista influenzato dalle informazioni condivise attraverso i social media.


\clearpage{\pagestyle{empty}\cleardoublepage}
