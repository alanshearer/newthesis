%%%%%%%%%%%%%%%%%%%%%%%%%%%%%%%%%%%%%%%%%%%%%%%%%%%%%%%%%%%
% Appendice A 3

\chapter{Project Charter}
\label{appendiceA}


\textbf{Nome del progetto}: @Scavi\\

\textbf{Breve descrizione dell’idea}\\
Il progetto @Scavi nasce dall'esigenza di fornire i visitatori del sito archeologico di Paestum di uno strumento che assista loro durante la visita degli scavi.
 
\textbf{Benefici attesi dalla sua realizzazione}\\
In questa sezione indicare perché la vostra iniziativa dovrebbe essere intrapresa (es. aumento del fatturato di X, riduzione costi di Y, accesso a un nuovo mercato con poteziale Z ecc.). Ove possibile cercate di essere ‘numerici’, anche i progetti più ‘astratti’ alla fine possono essere ricondotti ad un beneficio numerico. Se non ce la fate a fare questo, molto probabilmente la vostra idea non è sufficientemente ‘matura’ per partire e gli altri attori faticheranno a darvi il supporto necessario.

\textbf{Obiettivi del progetto}\\
Gli obiettivi del progetto descrivono che cosa il progetto deve ottenere al fine di soddisfare i benefici attesi. Un progetto può avere più obiettivi, è normale. Ogni attore può/deve enunciare che cosa si aspetta dalla vostra iniziativa. Gli obiettivi dovrebbero essere enunciati secondo il metodo S.M.A.R.T. (Specifico, Misurabile, ‘Arrivabile’, con Risorse e Tempi). Ad esempio ‘Progettare e mettere in produzione il prodotto XY, con un costo unitario minore di Z, entro Novembre 2011, utilizzando non più di 6 ricercatori e un budget inferiore a €800K.

\textbf{Attori Principali e loro ruolo}\\
Ogni progetto ha almeno un Cliente (chi paga e si attende i Benefici, e quindi li definisce nel DIP), un Utente (chi utilizzerà i risultati del progetto per realizzare i benefici attesi del Cliente. L’utente solitamente scrive le Specifiche delle cose da fare) e il Fornitore (colui in quale fa il lavoro, partendo dalle specifiche dell’utente). Ultimo per ordine ma non per importanza il Responsabile del Progetto, ovvero colui che ‘tiene le fila’ coordina le attività e si assicura che tutti gli attori siano sempre allineati con lo stato del progetto.

\textbf{Lista delle cose da fare}\\
Inizialmente in modo disorganico, ogni attore scrive tutte le cose che, secondo lui, servono per il progetto. Una volta che tutti hanno detto la loro, il Responsabile del Progetto facilita il raggruppamento delle cose da fare in ‘Pacchetti di lavoro’ che abbiano senso compiuto. Il risultato sembra un organigramma, nei cui primi livello compaiono i pacchetti di lavoro e in quelli successivi le attività per completarli. I tecnici chiamano questo prodotto WBS (Work Breakdown Structure).

\textbf{Analisi dei rischi}\\
Una volta note le cose da fare, occorre che tutti gli attori pensino a cosa ‘possa andare storto’ sia da un punto di vista del progetto (difficoltà nello svolgere alcune attività magari) che operativo (magari il prodotto del progetto mette a rischio il cliente – pensate per esempio ai Derivati che hanno fatto partire questa crisi finanziaria! Essi non sono altro che il risultato di un progetto, no?). i ricchi vanno classificati per Impatto e Probabilità, e quelli che hanno un impatto e una probabilità significativa vanno analizzati con cura e gestiti, aggiungendo altre attività nella lista delle cose da fare, e eventualmente altri costi nel budget .

\textbf{Esclusioni, Vincoli e Presupposti}\\
É fondamentale che tutti gli attori sappiano non solo le cose da fare, ma anche le cose che non sono comprese nel progetto, ma che sono necessarie per fare raggiungere al cliente i benefici attesi. Queste sono le Esclusioni (ad esempio il progetto fornisce un nuovo software che riduce i costi di produzione, ma non la formazione agli operatori. Senza la formazione gli operatori non usano il software e quindi non raggiungono il beneficio di riduzione costi. Qualcuno ci deve pensare! Ma non nel nostro progetto, il cliente dovrà far partire un progetto di formazione parallelo al nostro.)

I Vincoli invece possono essere di tipo economico, temporale, legislativo, tecnologico etc. Documentare i vincoli è fondamentale per sapere i gradi di libertà delle nostre decisioni, e per informare tutti gli attori degli stessi.

I Presupposti catturano tutte le cose che noi ‘diamo per scontate’ nel decidere le tempistiche, i costi e le attività del progetto. Ad esempio io posso dare per assodato (perché mi sto mettendo d’accordo ora) che l’utente dovrà rispondere alle domande del fornitore massimo entro 24 ore, oppure che il cliente si impegna a fornire una risorsa esperta per l’analisi, oppure ancora che il costo del denaro è del x,xx% e quindi il progetto ha senso economicamente. Importante: quando cambiano i presupposti, si deve rivedere questo documento, e aggiornare i piani temporali, economici e l’analisi di rischio.

\textbf{Piano logico-temporale}\\
Questo è il famoso ‘diagramma di GANTT’ in cui si legano logicamente le une alle altre le attività, insieme al cliente e fornitore si decidono i tempi e si crea quindi ‘il navigatore GPS’ del progetto.

\textbf{Piano dei costi}\\
Ogni attività nel piano logico temporale avrà ovviamente un costo (ore uomo, materiali, viaggi, etc.). La somma di tutte queste attività ci fornisce il Budget del Progetto

\textbf{Procedure per la reportistica e l’avanzamento lavori}\\
Come gli attori intendo comunicare tra loro? Cosa e in che modo desiderano sapere? Report di avanzamento lavori, incontri per la verifica delle attività e la risoluzione dei problemi etc. É importante mettersi d’accordo per evitare di sprecare tempo in comunicazioni ad-hoc, o al contrario in inutili riunioni-fiume o tonnellate di e-mail.

\textbf{Approvazione di tutti gli attori}\\
Una volta deciso quanto sopra, tutti gli attori approvano le ‘regole del gioco’ che hanno scritto insieme e incominciano a portare avanti le attività vere e proprie. In realtà molto lavoro è già stato fatto con il DIP e il progetto sta partendo decisamente col piede giusto!



%%%%%%%%%%%%%%%%%%%%%%%%%%%%%%%%%%%%%%%%%%%%%%%%%%%%%%

%\clearpage{\pagestyle{empty}\cleardoublepage}