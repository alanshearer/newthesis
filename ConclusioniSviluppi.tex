%%%%%%%%%%%%%%%%%%%%%%%%%%%%%%%%%%%%%%%%%%%%%%%%%%%%%%%%%%%
% Capitolo 1

\chapter*{Conclusioni e sviluppi futuri}
\label{ref:conclusioni}  \addcontentsline{toc}{chapter}{Conclusioni e sviluppi futuri}
In questa tesi è stata sviluppata un'applicazione mobile per la navigazione assistita dei beni culturali.
La tesi è stata sviluppata in ambito aziendale, all'interno del più generale contesto del distretto per i beni culturali DATABENC.\\
L'applicazione consente di ottenere informazioni georeferenziate sul bene culturale fisicamente più vicino, tramite un meccanismo di georeferenziazione ed un servizio che estrapola le informazioni rilevanti provenienti da una banca dati.\\
Tale applicazione è stata realizzata per gli smartphone basati su Windows Phone, sfruttando le tecnologie di sviluppo fortemente integrate fornite da Microsoft. 
L'innovazione della tesi è stata rappresentata dallo sviluppo di un'applicazione per una famiglia di dispositivi recentemente immessa sul mercato e dall'adozione di una metodologia di sviluppo agile che ha integrato e valorizzato le scelte tecnologiche effettuate in un contesto coerente.\\
Il server a cui i client accedono è stato realizzato tramite un approccio Cloud di tipo PaaS (Platform as a Service) che ha utilizzato il sistema Microsoft Azure.
L'applicazione è stata quindi testata sul campo, mostrando come le informazioni fornite si adeguino al contesto ed è stata richiesta a Microsoft la pubblicazione sul catalogo delle applicazioni.\\

Gli sviluppi futuri sono riconducibili a queste linee di attività:
\begin{itemize}
\item estensione delle funzionalità lato server (esempio sistema di recommendation);
\item estensione della base dati ad altri siti di interesse;
\item implementazione dell'app per altre famiglie di dispositivi mobili.
\end{itemize}


\clearpage{\pagestyle{empty}\cleardoublepage}
%%%%\clearpage{\pagestyle{empty}\cleardoublepage}